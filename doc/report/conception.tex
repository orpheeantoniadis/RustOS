%%%%%%%%%%%%%%%%%%%%%%%%%%%%%%%%%%%%%%%%%%%%%%%%%%%%%%%%%%%%%%%%%%
%%%%%%%%%%%%%%%%%%%%%%%%%%%%%%%%%%%%%%%%%%%%%%%%%%%%%%%%%%%%%%%%%%

\subsection{Environnement de développement}
La machine utilisée pour le développement du projet est un MacBook Pro avec un
processeur Intel à 3 GHz. Il a quand même fallut utiliser une machine virtuelle
(VMware) utilisant Linux (Ubuntu 16.04.4 LTS) pour la compilation. Ce choix a été
fait car il existe beaucoup plus de documentation sur l'implémentation de systèmes
d'exploitation sur Linux que sur Mac. Bien que Mac \acrshort{os} soit un système UNIX, les
exécutables générés sur cet environnement n'ont pas le même format que ceux générés
sur Linux qui sont au format \acrshort{elf}. Ceci rend le développement d'\acrshort{os} légèrement
différent sur Mac \acrshort{os}.

%%%%%%%%%%%%%%%%%%%%%%%%%%%%%%%%%%%%%%%%%%%%%%%%%%%%%%%%%%%%%%%%%%
%%%%%%%%%%%%%%%%%%%%%%%%%%%%%%%%%%%%%%%%%%%%%%%%%%%%%%%%%%%%%%%%%%

\subsection{Technologies}
\subsubsection{Nasm}
Bien que le système d'exploitation développé devait être sur Rust, certaines parties
ont du être faites en assembleur car étant trop bas niveau pour le Rust. Ces éléments
seront décrits plus loin dans ce document. Nasm a été  utilisé pour compiler le
code assembleur x86 en \acrshort{elf} 32-bits. Nasm produit des fichiers objets pouvant être
liés à d'autres fichiers objets afin de créer un exécutable. \\

\subsubsection{Rustup}
Rust sera décrit plus en détails dans un prochain chapitre. Ce qu'il faut savoir
est que Rust est distribué sous trois versions différentes. La version \textit{stable},
la version \textit{beta} et la version \textit{nightly}. La version \textit{nightly}
possède plus de fonctionnalités mais sa stabilité n'est pas garantie. Cette version
a été utilisée pendant le développement du projet et l'utilitaire Rustup a été utilisé
pour son installation. Cet utilitaire permet de simplifier l'installation de Rust
quand on souhaite une version différente de la dernière version stable de Rust. \\

\subsubsection{Cargo et Xargo}
Lors du développement d'un système d'exploitation type \textit{bare metal}, on souhaite
s'affranchir de toute dépendance à une librairie externe. Tout doit être refait depuis
le début. Le code est donc compilé sans la bibliothèque standard (std). Rust a tout
de même besoin d'une base pour être compilé. Cette base est fournie par la librairie
\mintinline{rust}{core}. Cette librairie est minimale et permet de ne définir que
les primitives de Rust. Pour gérer les dépendences d'un projet Rust, il est conseillé
d'utiliser le gestionnaire de paquets cargo. Le problème est que cargo ne permet
pas de lier la librairie \mintinline{rust}{core} à un projet. Heureusement, un 
autre utilitaire basé sur cargo existe et permet d'installer par défaut la librairie
\mintinline{rust}{core} pour des projets sans bibliothèque standard. Cet utilitaire
se nomme xargo et est utilisé pour compiler le code Rust en fichiers objets \\

\subsubsection{QEMU}
Le compilateur \acrshort{gcc} a été utilisé pour \textit{linker} les fichiers
objet générés par nasm et xargo. \acrshort{gcc} génère un fichier au format \acrshort{elf}.
Pour utiliser ce fichier comme un système d'exploitation \textit{bootable}, il faut
en faire une image \acrshort{iso} \textit{bootable}. Pour se faire, l'utilitaire
\mintinline{rust}{genisoimage} est utilisé, couplé au \textit{bootloader} \acrshort{grub}.
L'image \acrshort{iso} est finalement exécutée par la machine virtuelle QEMU.
QEMU est une machine virtuelle pouvant émuler une architecture. Pour ce projet,
l'architecture i386 a été choisie afin d'émuler un processeur Intel 32-bits.

%%%%%%%%%%%%%%%%%%%%%%%%%%%%%%%%%%%%%%%%%%%%%%%%%%%%%%%%%%%%%%%%%%
%%%%%%%%%%%%%%%%%%%%%%%%%%%%%%%%%%%%%%%%%%%%%%%%%%%%%%%%%%%%%%%%%%

\subsection{Architecture}